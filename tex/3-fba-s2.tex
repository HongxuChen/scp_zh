\subsection{例子和讨论}\label{sec:fba_eg_disscuss}
\todo{图3}展示了一个层状系统,系统中的不同节点有着截然不同的切片集合,这可能只有在FBA才能实现。顶层由$v_1,\ldots,v_4$组成,其结构类似于PBFT中$f=1$的情形,这意味着只要其它三个节点可达并正常工作,该系统就可以容忍一个拜占庭错误。节点$v_5,\ldots,v_8$组成中间层不相互依赖,而是依赖于顶层。中间层节点只需要两个顶层节点就可以形成切片。(假定顶层只有最多一个拜占庭错误,那么除非整个系统出错否则两个顶层节点不会同时出现错误。)节点$v_9$和$v_{10}$在叶子层,其切片由任意两个中间层节点组成。注意这里$v_9$和$v_{10}$可能选择不相交的切片集合,例如$\left\{v_5,v_6\right\}$和$\left\{v_7,v_8\right\}$;然而,两者都会间接依赖于顶层节点。

实际情形中,顶层可能包含来自各处的四到几十个广为人知并可信的金融机构。当顶层的大小增长时,可能不再有关于它的成员关系的准确一致性,但是大多数团体对顶层节点的意识将会重叠。另外,还可以想象很多的中间层,例如每个代表一个国家或者地理区域。

这种分层系统和域名间的网络路由系统十分类似。当今的网络是由独立的对等直连和网络对间的传输关系共同组成的。没有中央权威机构来指派或仲裁这些安排。然而这些成对的关系也已足够创建出实际意义上的第一层结构——网络服务提供商(ISP)~\cite{peer_isp2010}。	尽管英特网的可达性受防火墙影响,但\textit{传递}可达性几乎是完全的——例如,某个防火墙可能会阻塞纽约时报,但如果它允许Google访问,而Google能够访问纽约时报,那么纽约时报也间传递性地可达。传递可达性对网站来说或许是受限的设施,但是这对共识至关重要;等价的例子是Google仅当在纽约时报接受某个陈述的时候才接受该陈述。

如果我们把{\quorum}切片看成类似网络可达性,而把{\quorum}看成是传递可达性,那么网络近乎完全的传递可达性启示我们同样也可以利用FBA达成世界范围内的共识。在很多方面,共识比网际间的路由选择要容易得多。传输消耗资源且花费资金,但包含切片的过程仅仅要求检查数字签名。因此,FBA节点可以在包含切片的那端报告错误,相比较常见的在对等直连和传输约定场景见到的那样,这可以用更为相互依赖且冗余的方式来建立保守的切片。

另外,正如\todo{图4}中展示的那样,中心化共识对有环依赖结构也无能为力。这样一个环状结构不会被有意生成出来,然而当独立的节点选择它们自己的切片时,这可能导致整个网络最终被植入了环状依赖。更大的问题是,相比较传统的拜占庭协商来说,一个FBA的协议必须解决远为多样的{\quorum}结构。