\subsection{例子和讨论}\label{sec:fba_eg_disscuss}

\begin{figure}
\centering
\begin{tikzpicture}[thick,
  %box/.style={ultra thick,draw=gray,rounded corners, inner sep=.4cm},
%  box/.style={ultra thick,rounded corners,inner
%    sep=.3cm, fewshade=\maincolor!30!white},
  node/.append style={minimum width=4.5ex, inner sep=0},
  ]
\node[node] (v1) {$v_1$};
\node[node, right=.4cm of v1] (v2) {$v_2$};
\node[node, right=.4cm of v2] (v3) {$v_3$};
\node[node, right=.4cm of v3] (v4) {$v_4$};
\begin{scope}[on background layer]
\node[box,fit=(v1) (v4)] (bft) {};
\end{scope}
\node[anchor=west,text width=5cm,font={\small},outer sep=2mm] at (bft.0)
%
     {顶层:  切片是包含自身的$\{v_1,v_2,v_3,v_4\}$的三元子集};
%
\node[node, below=1cm of $(v1|-bft.south)$] (v5) {$v_5$};
\node[node, right=.4cm of v5] (v6) {$v_6$};
\node[node, right=.4cm of v6] (v7) {$v_7$};
\node[node, right=.4cm of v7] (v8) {$v_8$};
\begin{scope}[on background layer]
\node[box,fit=(v5) (v8)] (bftdep) {};
\end{scope}
\node[anchor=west, text width=5cm, font=\small,outer sep=2mm]
  at (bftdep.east) {
  中间层: 切片是自身+任意2个顶层节点};
\node[node, below=1cm of $(v6|-bftdep.south)$] (v9) {$v_9$};
\node[node, right=.4cm of v9] (v10) {$v_{10}$};
\begin{scope}[on background layer]
\node[box,fit=(v9) (v10)] (bftdepdep) {};
\end{scope}
\node[anchor=west,text width=5cm, font=\small, outer sep=2mm] at
(bftdepdep.east-|bftdep.east)
     {叶子层: 切片是自身 + 任意2个中间层节点};
\draw[transition] (bftdepdep) --
  node[right,pos=.45,anchor=west] {\footnotesize 2/4} (bftdep);
\draw[transition] (bftdep) --
  node[right,pos=.45,anchor=west] {\footnotesize 2/4} (bft);
\draw[transition] (bft) to[loop above,looseness=5]
  node[pos=.8,right,anchor=west] {\footnotesize 3/4} (bft);
\end{tikzpicture}\\
\caption{层状{\quorum}结构举例}
\label{fig:hierarchy}
\end{figure}

图\ref{fig:hierarchy}展示了一个层状系统,系统中的不同节点有着截然不同的切片集合,这可能只有在FBA才能实现。顶层由$v_1,\ldots,v_4$组成,其结构类似于PBFT中$f=1$的情形,这意味着只要其它三个节点可达并正常工作,该系统就可以容忍一个拜占庭错误。节点$v_5,\ldots,v_8$组成中间层不相互依赖,而是依赖于顶层。中间层节点只需要两个顶层节点就可以形成切片。(假定顶层只有最多一个拜占庭错误,那么除非整个系统出错否则两个顶层节点不会同时出现错误。)节点$v_9$和$v_{10}$在叶子层,其切片由任意两个中间层节点组成。注意这里$v_9$和$v_{10}$可能选择不相交的切片集合,例如$\left\{v_5,v_6\right\}$和$\left\{v_7,v_8\right\}$;然而,两者都会间接依赖于顶层节点。

实际情形中,顶层可能包含来自各处的四到几十个广为人知并可信的金融机构。当顶层的大小增长时,可能不再有关于它的成员关系的准确一致性,但是大多数团体对顶层节点的意识将会重叠。另外,还可以想象很多的中间层,例如每个代表一个国家或者地理区域。

这种分层系统和域名间的网络路由系统十分类似。当今的网络是由独立的对等直连和网络对间的传输关系共同组成的。没有中央权威机构来指派或仲裁这些安排。然而这些成对的关系也已足够创建出实际意义上的第一层结构——网络服务提供商(ISP)~\Citep{peer_isp2010}。	尽管英特网的可达性受防火墙影响,但\textit{传递}可达性几乎是完全的——例如,某个防火墙可能会阻塞纽约时报,但如果它允许Google访问,而Google能够访问纽约时报,那么纽约时报也间传递性地可达。传递可达性对网站来说或许是受限的设施,但是这对共识至关重要;等价的例子是Google仅当在纽约时报接受某个陈述的时候才接受该陈述。

如果我们把{\quorum}切片看成类似网络可达性,而把{\quorum}看成是传递可达性,那么网络近乎完全的传递可达性启示我们同样也可以利用FBA达成世界范围内的共识。在很多方面,共识比网际间的路由选择要容易得多。传输消耗资源且花费资金,但包含切片的过程仅仅要求检查数字签名。因此,FBA节点可以在包含切片的那端报告错误,相比较常见的在对等直连和传输约定场景见到的那样,这可以用更为相互依赖且冗余的方式来建立保守的切片。


\begin{figure}
\centering
\begin{tikzpicture}[thick]
\foreach \v in {1,2,...,6} {
  \path let \n1={int(150-\v*60)} in
  (\n1:1.75) node[circle,fewshade=\maincolor] (v\v) {$v_{\v}$}
  % edge[in=\n1-15,out=\n1+15,loop] ()
  ;
}
\foreach \x in {1,2,...,6} {
  \pgfmathparse{int(mod(\x,6)+1)}\edef\xx{\pgfmathresult}
  \draw[->,shorten >=.2pt,very thick,few-gray] (v\x) to[bend left=22] (v\xx);
}
\node[right={1em of $(v2.east)!.5!(v3.east)$}]
     {\small $\Q{v_i}=\left\{\{v_i,v_{(i\bmod6)+1}\}\right\}$};
\end{tikzpicture}\\
\caption{环形{\quorum}结构举例}
\label{fig:cycle}
\end{figure}

另外,正如图\ref{fig:cycle}中展示的那样,中心化共识对有环依赖结构也无能为力。这样一个环状结构不会被有意生成出来,然而当独立的节点选择它们自己的切片时,这可能导致整个网络最终被植入了环状依赖。更大的问题是,相比较传统的拜占庭一致性来说,一个FBA的协议必须解决远为多样的{\quorum}结构。