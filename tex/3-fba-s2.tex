\subsection{例子和讨论}
\todo{图3}展示了一个层状系统,系统中的不同节点有着截然不同的切片集合,这可能只有在FBA才能实现。顶层由$v_1,\ldots,v_4$组成,其结构类似于$PBFT$中$f=1$的情形,这意味着只要其他三个节点可达并正常工作,它可以容忍一个拜占庭故障。节点$v_5,\ldots,v_8$组成中间层并且不相互依赖,而是依赖于顶层。中间层的节点只要求两个顶层节点就可以形成切片。(假定顶层只有最多一个拜占庭故障,那么除非整个系统出错否则两个顶层节点不会同时出现故障。)节点$v_9$和$v_10$在叶子层,其切片由任意两个中间层节点组成。注意这里$v_9$和$v_10$可能选择不相交的切片集合,例如$\left\{v_5,v_6\right\}$和$\left\{v_7,v_8\right\}$;然而,两者都会间接依赖于顶层节点。

实际情形中,顶层可能包含来自各个地方的从四到几十个广为人知并可信的金融机构。当顶层的大小增长时,可能不再有关于它的成员关系的准确认可,但是将会有顶层成员间意识上很大程度的重叠。另外,我们可以想象很多的中间层,例如每个代表一个国家或者地理区域。

这种分层系统和域名间的网络路由系统十分类似。