\subsection{陈述确认}

两类被接受陈述的局限性都来源于这一事实:一个完好节点$v$可能会给一个已经被批准的陈述$a$投反对票。在反对$a$之后,它不会再投赞成票,这使得$v$将不可能批准$a$。为了给$v$提供一种在给$a$投反对票之后仍然能够批准$a$的方法,\textit{接受}的定义给出了一个基于{\vblock}的第二准则。但是这个第二准则弱于批准,对有{\quorum}交的被污染节点没有任何保障。

现在如果一个陈述$a$有``从来没有任何完好节点反对它''这样一个属性,那么我们没有必要去接受它。完好节点可以简单地批准$a$,而我们可以在作用于$a$之前要求它们这样做。我们成这样的陈述是不可驳斥的。

\begin{definition}[不可驳斥]
	当没有完好节点可以反对一个陈述$a$时,称这条陈述在FBAS中是\textbf{不可驳斥的}。
\end{definition}

定理\ref{thm:intact_cannot_accept_contradictory}告诉我们两个完好节点不可能接受相互冲突的陈述。因此,尽管一些节点可能会反对某个被完好节点接受的陈述$a$,对``一个完好节点接受了$a$''这一陈述是不可反驳的。这建议我们举行第二次投票来批准``一个完好节点接受了$a$''这一事实。

\begin{definition}[确认]
	在FBAS中一个{\quorum}$U_a$\textbf{确认}一个陈述$a$当且仅当$\forall v\in U_a$, $v$声称承认$a$。一个节点\textbf{确认}$a$当且仅当它在这样一个{\quorum}中。
\end{definition}

节点通过声称``$accept(a)$''的接受陈述$a$,这是``一个完好节点接受$a$''的缩略形式的陈述。一个良性行为的节点$v$仅当在接受了$a$之后才能投票赞成$accept(a)$,这是因为$v$不能假定其它的任何节点都是完好的。如果$v$本身是被污染的,$accept(a)$可能会失败,在这种情形投票赞成它可能会牺牲$v$的存活性,然而无论如何一个被污染的节点对存活性是没有任何保证的。

下一个定理表明节点能够依赖被确认陈述而不损失最佳安全性。定理\ref{thm:confirmed_stats_keep_liveness}则说明被确认陈述符合第\ref{sec:accept_not_enough_liveness}节中\textit{认可}的定义,这意味着节点可以依赖被确认陈述而不会破坏完好节点的存活性。

\begin{theorem}\label{thm:confirmed_stats_keep_safety}
	设$\langle\mybm{V},\mybm{Q}\rangle$是一个满足除$B${\quorum}可交性的FBAS,且假定$B$包含了所有的恶性行为节点。令$v_1$和$v_2$是不在$B$中的两个节点。设$a$和$\bar a$是相互冲突的陈述。则有,若$v_1$确认了$a$,则$v_2$不会确认$\bar a$。
\end{theorem}

\begin{proof}
	首先注意到$accept(a)$和$accept(\bar a)$相互冲突——没有一个良性行为的节点会同时投票赞成两者。更进一步,注意到$v_1$必须批准$accept(a)$以确认$a$。由定理\ref{thm:nodes_cannot_ratify_contracdictory},$v_2$不会批准$accept(\bar a)$因此不会确认$\bar a$。
\end{proof}

\begin{theorem}\label{thm:confirmed_stats_keep_liveness}
	如果一个含{\quorum}交的FBAS $\langle\mybm{V},\mybm{Q}\rangle$确认了一个陈述$a$,那么不论接下来会发生什么,一旦足够的消息被发送并处理了,每个完好节点将会接受并确认$a$。
\end{theorem}

\begin{proof}
	令$B$是被污染节点的$DSet$,并令$U_a\not \subseteq B$是一个{\quorum},一个完好节点通过它确认$a$。让在$U_a\backslash B$中的节点广播$accept(a)$。根据定义,任何节点$v$,不论它是如何投票的,在从一个{\vblock}集合中接受了$accept(a)$陈述之后将接受$a$。因此,这些消息会说服额外的节点去接受$a$。让这些额外节点反过来广播$accept(a)$直到到达这样一个时刻:不论进一步的通信如何,没有更多的完好节点会接受$a$。一旦这个过程完成,设$V^+$是已经接受$a$的所有完好节点集合,设$V^-=(\mybm{V}\backslash B)\backslash V^+$是那些不能接受$a$的完好节点集。为证明所有的完好节点都接受了$a$,我们必须证明$V^-=\emptyset$。

	由定理\ref{thm:quorum_subset_is_quorum},$U_a\backslash B$是$\langle\mybm{V},\mybm{Q}\rangle^{B}$中的一个{\quorum}。由于对于任何$v\in V^-$来说$V^+$都不是{\vblock}的,那么由定理\ref{thm:quorum_availability_vs_vblocking} 要么$V^-=\emptyset$要么$V^-$是$\langle\mybm{V},\mybm{Q}\rangle^{B}$中的一个{\quorum}。后一种情形导致这样一个矛盾:由于$V^-$只包含完好的(因此也是良性行为的)节点,它们当中的没有能够在首先真正接受$a$的情况下声明$accept(a)$,这意味着$U_a\backslash B\cap V^-=\emptyset$。然而这是不可能的,因为除$B$(一个$DSet$){\quorum}可交性告诉我们它的反面:$U_a\backslash B\cap V^-\neq\emptyset$。剩下只可能是$V^-=\emptyset$。一旦每个节点都接受了$a$,所有的都会投票赞成确认$a$。因为完好节点构成了一个{\quorum},这些投票将会成功。
\end{proof}


\begin{figure}
\centering
\begin{tikzpicture}[thick,
    node/.style={draw,ellipse,inner xsep=-1.5mm,align=center,text
      width=2.2cm,
      inner ysep=0pt,
      text depth=1ex,text height=1em,fewshade=\maincolor},
    transition/.style={->,very thick, align=center,shorten
      >=0pt,shorten <=-3pt,font=\small\openup-.5\jot,draw=few-gray},
  ]
\node[node] (N) at (0,0) {未提交};
\node[node] (P) at (3,1) {已赞成~$a$};
\node[node] (A) at (6,1) {已接受~$a$};
\node[node] (C) at (9,1) {已确认~$a$};
\node[node] (nP) at (3,-1) {已赞成~$\na$};
\begin{scope}[on background layer]
%\draw[transition] (P.60) to[out=60,in=180]
%    (4.5,2) -- 
%    (7.5,2)
%    to[out=0,in=120]
%    node[above,text depth=.5pt,pos=0] {have signed\\ evidence}
%    (C.120);
\draw[transition] (P.60) to[out=60,in=180]
    (4.5,2) 
    to[out=0,in=120,pos=0]
    node[above,text depth=.5pt,align=left] {\quorum
      赞成\\或接受$a$}
    (A.120);
%\draw[transition] (A.60) to[out=60,in=180]
%    node[above,pos=1,text depth=0pt] {ratify $\confirm a$} (7.5,2)
%    to[out=0,in=120] (C.120);
\draw[transition] (A.60) to[out=60,in=180]
    (7.5,2) to[out=0,in=120]
    node[above,pos=.01,text depth=0pt,align=left]
    {\quorum 确认 $a$}  (C.120);
%\draw[transition] (A.-60) to[out=-60,in=180] (7.5,0)
%    to[out=0,in=-120]
%    node[below right=0pt,inner sep=0,pos=0,text depth=0pt]
%    {ratify $\confirm{a}$}
%    (C.-120);
\draw[transition] (N) to[out=90,in=180,looseness=.9]
     node[above,align=center] {$a$是有效的} (P);
\draw[transition] (N) to[out=-90,in=180,looseness=.9] (nP);
\draw[transition,-] (P.-60) to[out=-60,in=180] (4.5,0);
\draw[transition] 
       (nP.60) to[out=60,in=180] (4.5,0)
       to[out=0,in=-120]
       node[below right,inner sep=0,pos=.1,align=left]
            {{\vblock}集\\[1pt] 接受 $a$}
       (A.-120);
\end{scope}
\end{tikzpicture}
\caption{一个被接受的陈述$a$在节点$v$处可能的状态}
\label{fig:node-states}
\end{figure}

{图\ref{fig:node-states}}总结了一个完好节点为了确认$a$可以采用的路径。在不知情的状态下,$v$可能会赞成$a$或者与之冲突的$\bar a$。如果$v$投票赞成$\bar a$,它之后不会再赞成$a$;但是如果一个{\vblock}集合接受了$a$那么$v$却可以接受$a$。接下来确认消息的{\quorum}允许$v$去确认$a$,由定理\ref{thm:confirmed_stats_keep_liveness}这意味着系统认可$a$。