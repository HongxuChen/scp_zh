\section{联邦拜占庭一致性系统}\label{sec:fba}

本节介绍联邦拜占庭一致性(Federated Byzantine Agreement, FBA)模型。和非联邦一致性协议一样,FBA需要解决更新重复的\footnote{译注:副本(replica)是分布式系统中为数据或服务提供的冗余。}状态的问题,例如事务分类帐或认证树。通过认可应用什么样的更新,节点避免了冲突而不可协调的状态。我们通过一个唯一的{\slot}(slot)来识别更新,这使得我们可以推断出更新依赖。例如,在一个顺序记录的日志中{\slot}可以是连续标号的位置。

FBA系统运行一个可以确保节点认可{\slot}的内容的共识协议。节点$v$在它已经安全地更新了{\slot}$i$所依赖的所有{\slot}之后可以在{\slot}$i$处安全地更新$x$;另外,它相信所有正确工作的节点最终将会认可{\slot}$i$处的更新$x$。外部系统会以不可逆的方式对具体的值作出反应,因此一个节点不能够在之后改变这些值。

FBA系统的一大挑战是恶意的团体可能参与多次并在数量上超过诚实节点。因此,传统的基于占优数量的{\quorum}机制无法适用。取而代之的是,FBA采用一种分布式的方式来决定{\quorum},这是通过每个节点选择所谓的``{\quorum}切片''来实现的。接下来的子章节讨论了一些例子。最后,我们定义一个共识协议应当希望满足的关键属性:安全性和完整性的。

\subsection{{\quorum}切片}

在一个共识协议中,节点交换消息来对关于{\slot}的陈述进行断言。我们假设这些断言无法被伪造——如果节点通过公钥命名并且数字签名了这些消息那么这是可以保证的。当一个节点侦听到足够多的节点断言了某个陈述之后,它将假定不会有工作节点否定这一陈述。我们称这样的一个足够的集合为\textbf{{\quorum}切片},或简称\textbf{切片}。为了允许在节点失败的情形下系统仍然能够推进,一个节点可能有多个切片,它的任意一个切片都足够让它相信某个陈述。从较高层次来看,一个FBA系统由松散的节点邦联组成,其中的每个节点包含一个或多个切片。更形式化地讲:

\begin{definition}[FBAS]
	一个联邦拜占庭协商系统,或简称\textit{FBAS},是一个二元组$\langle\mybm{V}, \mybm{Q}\rangle$,它包含节点集合$\mybm{V}$和{\quorum}函数$\mybm{Q}:\mybm{V}\rightarrow2^{2^{\mybm{V}}}\backslash\left\{\emptyset\right\}$,后者用于指定每个节点的一个或多个切片,这里一个节点属于所有它自己的切片——即,$\forall v\in\mybm{Q},\forall q\in{\mybm{Q}(v)},v\in q$(注意$2^X$指的是$X$的幂集)。
\end{definition}

\begin{definition}[\quorum]
	FBAS$\langle\mybm{V},\mybm{Q}\rangle$中的节点集合$U\subseteq\mybm{V}$是一个\textbf{\quorum}当且仅当$U$包含每个节点的一个切片——即,$\forall v\in U, \exists q \in \mybm{Q}(v), s.t. q\subseteq U$。
\end{definition}

{\quorum}是足够达成一致的节点集合。切片是说服某一特定节点认可的{\quorum}子集。一个{\quorum}切片可能小于{\quorum}。考虑\todo{图2}中的四节点系统,每个节点包含单一切片而箭头指向切片中的其他成员。节点$v_1$的切片$\left\{v_1,v_2,v_3\right\}$足以说服$v_1$认可。但是$v_2$和$v_3$的切片包含$v_4$,这意味着$v_2$和$v_3$都不能在没有$v_4$同意的情况下断言某个陈述。因而,没有$v_4$的参与就不可能达成一致,而唯一的包含$v_1$的{\quorum}是所有节点的集合$\left\{v_1,v_2,v_3,v_4\right\}$。

传统非联邦共识要求所有的节点接受相同的切片,即$\forall v_1, v_2, \mybm{Q}(v_1)=\mybm{Q}(v_2)$。因为任意的切片都足够使得所有的节点相信所有节点的某一个陈述,中心化系统不加区分切片和{\quorum}。缺点在于成员关系和{\quorum}必须事先规定好,妨碍了开放式成员关系和去中心化控制。传统的系统,例如PBFT~\cite{Castro:1999:PBFT},通常有$3f+1$个节点,它们中的任意$2f+1$个节点组成一个{\quorum}。这里$f$是拜占庭故障的最大值——意味着节点的随意行为都能保证系统能够存活。

本文中介绍的FBA一般化了中心化共识以使得它可以容纳更广泛的环境。FBA的关键不同在于每个节点选择它自己的{\quorum}切片集合$\mybm{Q}(v)$。从而系统范围内的{\quorum}由每个节点的独立决策产生。节点可以根据任意准则选择切片,例如节点的信誉或财务安排。在一些环境下,可能对任何节点来说记录整个系统的所有节点集合${\mybm{V}}$是不现实的,然而仍然可以达成共识。
\subsection{例子和讨论}
\todo{图3}展示了一个层状系统,系统中的不同节点有着截然不同的切片集合,这可能只有在FBA才能实现。顶层由$v_1,\ldots,v_4$组成,其结构类似于PBFT中$f=1$的情形,这意味着只要其他三个节点可达并正常工作,该系统就可以容忍一个拜占庭错误。节点$v_5,\ldots,v_8$组成中间层不相互依赖,而是依赖于顶层。中间层节点只需要两个顶层节点就可以形成切片。(假定顶层只有最多一个拜占庭错误,那么除非整个系统出错否则两个顶层节点不会同时出现错误。)节点$v_9$和$v_{10}$在叶子层,其切片由任意两个中间层节点组成。注意这里$v_9$和$v_{10}$可能选择不相交的切片集合,例如$\left\{v_5,v_6\right\}$和$\left\{v_7,v_8\right\}$;然而,两者都会间接依赖于顶层节点。

实际情形中,顶层可能包含来自各处的四到几十个广为人知并可信的金融机构。当顶层的大小增长时,可能不再有关于它的成员关系的准确一致性,但是大多数团体对顶层节点的意识将会重叠。另外,还可以想象很多的中间层,例如每个代表一个国家或者地理区域。

这种分层系统和域名间的网络路由系统十分类似。当今的网络是由独立的对等直连和网络对间的传输关系共同组成的。没有中央权威机构来指派或仲裁这些安排。然而这些成对的关系也已足够创建出实际意义上的第一层结构——网络服务提供商(ISP)~\cite{peer_isp2010}。	尽管英特网的可达性受防火墙影响,但\textit{传递}可达性几乎是完全的——例如,某个防火墙可能会阻塞纽约时报,但如果它允许Google访问,而Google能够访问纽约时报,那么纽约时报也间传递性地可达。传递可达性对网站来说或许是受限的设施,但是这对共识至关重要;等价的例子是Google仅当在纽约时报接受某个陈述的时候才接受该陈述。

如果我们把{\quorum}切片看成类似网络可达性,而把{\quorum}看成是传递可达性,那么网络近乎完全的传递可达性启示我们同样也可以利用FBA达成世界范围内的共识。在很多方面,共识比网际间的路由选择要容易得多。传输消耗资源且花费资金,但包含切片的过程仅仅要求检查数字签名。因此,FBA节点可以在包含切片的那端报告错误,相比较常见的在对等直连和传输约定场景见到的那样,这可以用更为相互依赖且冗余的方式来建立保守的切片。

另外,正如\todo{图4}中展示的那样,中心化共识对有环依赖结构也无能为力。这样一个环状结构不会被有意生成出来,然而当独立的节点选择它们自己的切片时,这可能导致整个网络最终被植入了环状依赖。更大的问题是,相比较传统的拜占庭协商来说,一个FBA的协议必须解决远为多样的{\quorum}结构。
\subsection{安全性和存活性}\label{sec:fba-safe-live}

我们把节点分为\textit{良性行为的}和\textit{恶性行为的}。一个良性行为的节点选择可感知的{\quorum}切片(将在第\ref{sec:resilience}节深入讨论)并且遵守规则,这包括最终回应所有的请求。一个恶性行为的节点不是这样。恶性行为的节点受拜占庭错误的影响,这意味着它们可能会有随意的行为。例如,一个恶性节点可能会被入侵而它的所有者可能恶意地修改软件,或者它可能已经崩溃。

拜占庭协商的目标是要确保良性行为的节点要在恶性行为节点存在的情形下对外界表现出相同的值。这个目标包含两个方面。首先,我们希望防止节点分叉并对同一{\slot}向外界显示不同的值。其次,我们需要确保节点的确会得到值,而不是在某个死的状态下阻塞,这种情形下的共识变得不再可能。我们为这些属性引入下面两个术语:

\begin{definition}[安全性]
	FBAS中的一组节点集合,如果当中的任意两个节点都不会向外界表现出不同的值,那么它享有{\textbf{安全性}}。
\end{definition}

\begin{definition}[存活性]
	在FBAS中,如果给予合适的消息发送和时间选择节点就能够向外界表现出新值的话,那么该节点享有{\textbf{存活性}}。
\end{definition}

我们称这些既有安全性又有存活性节点是\textit{正确的}。不正确的节点被称为{\textit{有错误的}}。一个恶性行为的节点是有错误的,然而一个良性行为的节点也可能是有错误的——它可能无限等待某恶性行为节点的消息,或者更为严重的是它的状态被恶性行为节点用错误的信息给破坏了。

\todo{图5}强调了这样一种可能的节点错误。左边是拜占庭错误——恶性行为的节点。右边是两类良性行为的、但有错误的节点。不具有存活性的节点被标成了\textit{阻塞的},而那些不具有安全性的节点被标成了\textit{分叉的}。破坏安全性的攻击严格地比破坏存活性的攻击更为强大。因此我们把分叉节点归为阻塞节点的子集。

我们对\textit{存活性}的定义是比较弱的,这是因为它承认\textit{持久性优先抢占},在这样一种状态下总是可能达成共识,但是网络持续地用错误的方式延迟或调整重要信息的顺序来阻碍共识的形成。持久性优先抢占在一个纯异步、确定的、并能够免受节点错误干扰的系统里面是不可避免的~\cite{Fischer:1985:IDC:3149.214121}。幸运的是,抢占是暂时的。它并不意味着节点错误,因为系统可能会在任何时刻恢复。协议可以通过随机性(接入节点总是选择随机性的候选值直到有足够多节点正好选择了同一个节点~\cite{Ben-Or:1983:AFC:800221.806707,Bracha:1985:ACB:4221.214134})或对消息延迟的现实情况下的假设~\cite{Dwork:1988:CPP:42282.42283}来缓解这个问题。当希望限制执行时间时后一类协议更为实用。当然,结束性要求(termination)而不是安全性要求依赖于消息超时设置。


