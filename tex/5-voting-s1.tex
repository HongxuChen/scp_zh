\subsection{开放式成员关系下的投票系统}

在拜占庭一致性系统中,一个正确的节点仅会在它知道其他的正确节点永远不会接受一个和$a$相冲突的陈述的情形下接受$a$。大多数协议为了这一目标采用了选举。良性行为节点仅当在$a$是有效的时候才会给$a$投赞成票。良性行为节点也不会改变它们的投票结果。因此,在一个中心化拜占庭一致性中,如果绝大多数的良性节点(即{\quorum})投票赞成$a$的话,那接受$a$是安全的。当一个陈述接受到了必要的投票之后,我们称这样的陈述是\textit{被批准的}。

在一个联邦的环境中,我们必须调整选举机制以让他适应开放式成员关系。一个不同之处是{\quorum}不再对应于大多数良性行为节点。然而,多数性要求首先用于确保良性节点的{\quorum}交,在第\ref{sec:quorum_intersect}节我们已经把它调整成了FBA。开放成员关系还暗示了另一个问题需要解决:节点必须在一次选举过程中发现哪些节点组成了一个{\quorum}。为了实现{\quorum}发现,一个协议可以包含来自$v$的所有消息$\mybm{Q}(v)$,或提供其他一些办法来向$v$查询$\mybm{Q}(v)$。

\begin{definition}[投票]
	一个节点$v$\textbf{投票}赞成一个(抽象)陈述$a$当且仅当
	\begin{enumerate}
		\item $v$断言$a$是有效的,并且和其他$v$所接受的陈述相一致;且
		\item $v$断言$v$从未\textit{投票反对}$a$---即,投票赞成一个和$a$相冲突的陈述---且$v$承诺在未来绝不会给$a$投反对票。
	\end{enumerate}
\end{definition}

\begin{definition}[批准]
	一个{\quorum}$U_a$\textbf{批准}一个陈述$a$当且仅当每个$U_a$中的成员都投票赞成$a$。一个节点\textbf{批准}$a$当且仅当$v$是一个批准$a$的{\quorum}$U_a$的元素。
\end{definition}

\begin{theorem}\label{th4}
	两个冲突的陈述$a$和$\bar a$不会在一个在{\quorum}交且不含恶性行为节点的FBAS中同时被批准。
\end{theorem}

\begin{proof}
	反证法。假定{\quorum}$U_1$批准了$a$而{\quorum}$U_2$批准了$\bar a$。根据{\quorum}可交性,$\exists v\in U_1\cap U_2$。这样的节点$v$必然已经非法地投票赞成了$a$和$\bar a$,和没有恶性行为节点的假设想冲突。
\end{proof}

\begin{theorem}\label{th5}
	令$\langle\mybm{V},\mybm{Q}\rangle$是一个有除$B${\quorum}可交性的FBAS系统并假定$B$包含恶性行为的节点。设$v_1$和$v_2$是两个不在$B$中的节点,$a$和$\bar a$是相互冲突的陈述。则有,如果$v_1$批准了$a$那么$v_2$不会批准$\bar a$。
\end{theorem}

\begin{proof}
	反证法。假设$v_1$批准了$a$且$v_2$批准了$\bar a$。由定义可得,存在一个包含$v_1$的{\quorum}$U_1$批准了$a$,存在一个包含$v_2$的{\quorum}$U_2$批准了$\bar a$。由定理\ref{th1},由于$U_1\backslash B\neq \emptyset$且$\U_2\backslash B\neq \emptyset$。两者必是$\langle\mybm{V},\mybm{Q}\rangle^{B}$,这意味着他们各自在$\langle\mybm{V},\mybm{Q}\rangle^{B}$批准了$a$和$\bar a$。然而$\langle\mybm{V},\mybm{Q}\rangle^{B}$具有{\quorum}可交性且不含恶性行为节点,由定理\ref{th4}可知$a$和$\bar a$不可能同时被批准。
\end{proof}

\begin{theorem}\label{th6}
	一个具有{\quorum}交的FBAS中的两个完整节点不可能批准相互冲突的陈述。
\end{theorem}

\begin{proof}
	设$B$是受损的节点。由定理\ref{th3},$B$是一个$DSet$。由$DSet$的定义,$\langle\mybm{V},\mybm{Q},\rangle$有除$B${\quorum}可交性。由定理\ref{th5}可得不在$B$中的节点不可能批准冲突的陈述。
\end{proof}