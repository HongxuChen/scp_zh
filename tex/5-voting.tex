\section{联邦选举系统}

本节开发了一套可供FBAS中的节点对某个陈述认可的联邦投票技术。从高层次来看,对某个陈述$a$认可的过程涉及节点交换两个集合的消息。首先,节点为$a$投票。其次,如果投票成功,节点\textit{确认}$a$,在第一次投票成功的基础上有效地举行第二次投票。

在每个节点看来,两个回合的消息把对陈述$a$的认可分成了三个阶段:不知道的,认可的和确认的。至少,$a$的状态对一个节点来说是一无所知的---$a$可能最终为真,为假,甚或\textit{卡在}了一个长期不确定的状态上。如果第一次投票成功,$v$可能会\textit{接受}$a$。两个完整的节点从来不会接受相互矛盾的成熟,因此如果$v$是完整的并且接受了$a$那么$a$就不可能是假的。

然而由于两种原因,$v$接受$a$并不能说明$a$是真的。首先,$v$接受$a$这一事实并不代表所有完整的节点都能接受。其次,如果$v$是被污染的,那么接受$a$不能说明任何问题---$a$可能在完整节点处被认为是假的。然而即使$v$被污染了---$v$并不知道---系统仍然可以有良性行为节点的{\quorum}可交性,在这种情形下为了最优安全性,$v$需要对$a$的断言的更高的保证。举行第二次选举用来解决这两个问题。如果第二次选举成功,$v$转向到\textit{已确认}状态,这时它可以最终认为$a$是真并且作用于$a$上。

下面子章节阐述了联邦选举过程的细节。因为选举并没有去除被阻塞的陈述的可能性,第\ref{sec:vote_stuck}节讨论如何处理它们。在第\ref{sec:scp}节里我们将联邦选举传统转化成一个共识协议,它可以避免完整节点的被阻塞槽的可能性。

\subsection{开放式成员关系下的投票系统}

在拜占庭协商系统中,一个正确的节点仅会在它知道其它的正确节点永远不会接受一个和$a$相冲突的陈述的情形下接受$a$。大多数协议为实现这一目标采用了投票方式。良性行为节点仅当在$a$有效时才会给$a$投赞成票。良性行为节点也从不改变它们的投票结果。因此,在一个中心化拜占庭协商中,如果绝大多数的良性节点(即{\quorum})投票赞成$a$的话,那么接受$a$是安全的。当一个陈述收到必要的赞成票之后,我们称这样的陈述是\textit{被批准的}。

在一个联邦的环境中,我们必须调整选举机制以让它适应开放式成员关系。一个不同之处是{\quorum}不再对应于大多数良性行为节点。然而,多数性要求首先用于确保良性节点的{\quorum}交,在第\ref{sec:quorum_intersect}节我们已经调整它使之适用于FBA。开放成员关系还暗示了另一个问题需要解决:节点必须在选举过程中发现哪些节点组成了一个{\quorum}。为了发现{\quorum},一个协议在所有来自$v$的消息中包含$\mybm{Q}(v)$,或提供其它一些办法来向$v$查询$\mybm{Q}(v)$。

\begin{definition}[投票]
        一个节点$v$\textbf{投票}赞成一个(抽象)陈述$a$当且仅当
        \begin{enumerate}
                \item $v$断言$a$是有效的,并且和其它已被$v$接受的陈述相一致;且
                \item $v$断言$v$从未\textit{投票反对}$a$——即,投票赞成一个和$a$相冲突的陈述——且$v$承诺在未来决不会给$a$投反对票。
        \end{enumerate}
\end{definition}

\begin{definition}[批准]
        一个{\quorum}$U_a$\textbf{批准}一个陈述$a$当且仅当每个$U_a$中的成员都投票赞成$a$。一个节点\textbf{批准}$a$当且仅当$v$是一个批准$a$的{\quorum}$U_a$中的成员。
\end{definition}

\begin{theorem}\label{thm:cannot_ratify_contradictory}
        两个冲突的陈述$a$和$\bar a$不会在一个在{\quorum}交且不含恶性行为节点的FBAS中同时被批准。
\end{theorem}

\begin{proof}
        反证法。假定{\quorum}$U_1$批准了$a$而{\quorum}$U_2$批准了$\bar a$。根据{\quorum}可交性,$\exists v\in U_1\cap U_2$。这样的节点$v$必然已经非法地投票赞成了$a$和$\bar a$,和没有恶性行为节点的假设相冲突。
\end{proof}

\begin{theorem}\label{thm:nodes_cannot_ratify_contracdictory}
        令$\langle\mybm{V},\mybm{Q}\rangle$是一个有除$B${\quorum}可交性的FBAS系统并假定$B$包含恶性行为的节点。设$v_1$和$v_2$是两个不在$B$中的节点,$a$和$\bar a$是相互冲突的陈述。则有,如果$v_1$批准了$a$那么$v_2$不会批准$\bar a$。
\end{theorem}

\begin{proof}
        反证法。假设$v_1$批准了$a$且$v_2$批准了$\bar a$。由定义可得,存在一个包含$v_1$的{\quorum}$U_1$批准了$a$,存在一个包含$v_2$的{\quorum}$U_2$批准了$\bar a$。由定理\ref{thm:quorum_subset_is_quorum},由于$U_1\backslash B\neq \emptyset$且$U_2\backslash B\neq \emptyset$。两者必是$\langle\mybm{V},\mybm{Q}\rangle^{B}$,这意味着它们各自在$\langle\mybm{V},\mybm{Q}\rangle^{B}$批准了$a$和$\bar a$。然而$\langle\mybm{V},\mybm{Q}\rangle^{B}$具有{\quorum}可交性且不含恶性行为节点,由定理\ref{thm:cannot_ratify_contradictory}可知$a$和$\bar a$不可能同时被批准。
\end{proof}

\begin{theorem}\label{thm:intact_cannot_ratify_contracdictory}
        一个具有{\quorum}交的FBAS中的两个完整节点不可能批准相互冲突的陈述。
\end{theorem}

\begin{proof}
        设$B$是被污染的节点集。由定理\ref{thm:befouleds_are_dset},$B$是一个$DSet$。由$DSet$的定义,$\langle\mybm{V},\mybm{Q},\rangle$有除$B${\quorum}可交性。由定理\ref{thm:nodes_cannot_ratify_contracdictory}可得不在$B$中的节点不可能批准冲突的陈述。
\end{proof}
\subsection{阻塞集}
在中心化共识中,存活性是系统的一个要么都有要么都没有的属性。要么一个全体一致的良性行为{\quorum}存在,要么恶性行为节点会阻塞系统的其他部分接受新的陈述。在FBA中则不然,存活性可能在不同的节点间并不相同。例如,在\todo{图3}的分层{\quorum}例子中,如果中间层的$v_6$, $v_7$, $v_8$崩溃了,叶子层将会被阻塞,而顶层节点及$v_5$将继续具有存活性。

仅当$\mybm{Q}(v)$至少包含一个仅由正确节点组成的{\quorum}切片时,一个FBA协议可以保证对一个节点$v$的存活性。如果错误节点集合$B$至少包含每个$v$的切片的一个成员的话那么$B$会破坏这一属性。我们称集合$B$是一个{\vblock},因为它会阻塞$v$的进展。

\begin{definition}[{\vblock}]
        设$v\in \mybm{V}$是FBAS $\langle\mybm{V},\mybm{Q}\rangle$中的一个节点。集合$B\subseteq \mybm{Q}$是{\vblock}的当且仅当它和$v$的每个切片都有交集——即,$\forall q\in \mybm{Q}(v),q\cap B\neq \emptyset$。
\end{definition}

\begin{theorem}\label{th7}
        设$B\subseteq \mybm{V}$是FBAS $\langle\mybm{V},\mybm{Q}\rangle$的一个节点集合。$\langle\mybm{V},\mybm{Q}\rangle$具有除$B${\quorum}可达性当且仅当对任何$\mybm{V}\backslash B$来说$B$不是{\vblock}。
\end{theorem}

\begin{proof}
        ``$\forall v\in \mybm{V}\backslash B$, $B$不是{\vblock}''和``$\forall v\in\mybm{V}\backslash B, \exists q\in \mybm{Q}(v)$使得$q\subseteq \mybm{V}\backslash B$''等价。由{\quorum}的定义,后者成立当且仅当$\mybm{V}\backslash B$是一个群体或者$B=\mybm{V}$,亦即\textit{除$B${\quorum}可达性}的严格定义。
\end{proof}

作为结论,对任何完好的节点$v$来说,被污染节点的$DSet$不是{\vblock}的。
\subsection{接受陈述}
\subsection{接受是不够的}
不幸的是,假定被接受陈述的正确性在一个联邦共识协议中带来的安全性和存活性不是最优的。我们依次来讨论安全性和存活性。给定一些情景,我们然后讨论为什么在FBA中的问题比中心化拜占庭一致性中的更为棘手。

\subsubsection{安全性}\label{sec:voting_safety}
考虑一个仅包含唯一的、全体一致同意的{\quorum}的FBAS $\langle\mybm{V},\mybm{Q}\rangle$——即,$\forall v, \mybm{Q}(v)=\left\{\mybm{V}\right\}$。这应当是一个安全性的保守选择——一个节点只有在每个节点都同意之后才会做一件事。然而由于对任一$v$来说每个节点都是{\vblock}的,任何节点都能够独立地说服其它节点接受随意的陈述。

问题在于只有在完好节点中被接受陈述才是安全的。但正如第\ref{sec:quorum_intersect}讨论的那样,保证正确性的唯一的必要条件是良性行为节点的{\quorum}交,这有可能在一些良性行为节点被污染的情形下也会成立。特别地,当$\mybm{Q}(v)=\mybm{V}$时,仅有的$DSet$是$\emptyset$和$\mybm{V}$,这意味着任何一个节点错误将污染整个系统。然而,除$B\subseteq\mybm{V}${\quorum}可交性仍然成立。

\subsubsection{存活性}\label{sec:accept_not_enough_liveness}
另一个被接受陈述的局限性是,其它的完好节点可能不会接受它们。对存活性来说,这种可能性使得依赖于被接受的陈述是有问题的。如果一个节点由于它接受了一个陈述而作用于该陈述上,其它的节点可能无法以类似的方式处理该陈述。

\begin{figure}
\centering
\begin{tikzpicture}[thick,
    %node/.style={fewshade=\maincolor,circle,align=center,inner sep=0mm}
    node/.append style={align=center,inner sep=1pt},
  ]
\node[node] (v1) at (0,0)
     { $v_1$
       \\ 赞成 $a$ \\     接受 };
\node[node,right=.4cm of v1] (v2)
     { $v_2$ \\ 赞成 $a$ \\ \phantom{接受}};
\node[node,right=.4cm of v2] (v3)
     { \phantom{$v_3$} \\ \phantom{赞成 $a$} \\ \phantom{接受}};
\node[node,right=.4cm of v3] (v4)
     { $v_4$ \\ 赞成 $\na$ \\ \phantom{接受}};
\node[draw=few-\maincolor-bright,line cap=round,line width=1ex,
  opacity=.9,fit={(v3)},
  cross out,inner sep=-2mm] {};
\node[node,fill=none] at (v3) { $v_3$ \\ \textbf{赞成 $\bm{a}$} \\ };
\begin{scope}[on background layer]
\node[box,fit=(v1) (v4)] (bft) {};
\end{scope}
\draw[transition] (bft) to[loop above,looseness=5]
  node[pos=.8,right,anchor=west] {\footnotesize 3/4} (bft);
\node[anchor=base east, font={\small},align=left]
at ([yshift=4pt] bft.north east) {切片是包含自身的\\~3个节点};
%\node[overlay,right=4mm of v2, text width=18ex,font=\small]
%{ $\forall v\in\{v_1,\ldots,v_4\}$,
%  $\Q v$ is all sets of 3 nodes containing $v$.};
\node[anchor=south east,overlay] at ([xshift=-1ex] bft.south west) {a)};
\end{tikzpicture}\smallskip\\
\begin{tikzpicture}[thick,
    %node/.style={fewshade=\maincolor,circle,align=center,inner sep=0mm}
    node/.append style={align=center,inner sep=1pt},
  ]
\node[node] (v1) at (0,0)
     { \phantom{$v_1$}
       \\ \phantom{赞成 $a$} \\
       \phantom{接受} };
\node[node,right=.4cm of v1] (v2)
     { $v_2$ \\ 赞成 $a$ \\ \phantom{接受}};
\node[node,right=.4cm of v2] (v3)
     { $v_3$ \\ \textbf{赞成 $\bm{\na}$} \\ \textbf{接受} };
     %\setbox0=\hbox{accept}\hbox to\wd0{\hss \hss} };
\node[node,right=.4cm of v3] (v4)
     { $v_4$ \\ 赞成 $\na$ \\ \phantom{接受}};
\node[draw=few-\maincolor-bright,line cap=round,line width=1ex,
  opacity=.9,fit={(v1)},
  cross out,inner sep=-2mm] {};
\node[node,fill=none] at (v1) { $v_1$ \\ 赞成 $a$ \\ \textbf{赞成
    $\bm{\na}$}};
\begin{scope}[on background layer]
\node[box,fit=(v1) (v4)] (bft) {};
\end{scope}
\draw[transition] (bft) to[loop above,looseness=5]
  node[pos=.8,right,anchor=west] {\footnotesize 3/4} (bft);
\node[anchor=south east,overlay] at ([xshift=-1ex] bft.south west) {b)};
\end{tikzpicture}

\caption{$v_2$ 当$v_2$没有看到加粗消息时$v_2$无法区分的两种情形}
\label{fig:unresolvable}
\end{figure}

考虑图\ref{fig:unresolvable}a的情形,其中$v_3$在帮助批准$v_1$并接受陈述$a$之后崩溃了。尽管$v_1$接受了$a$,$v_2$和$v_4$却不能。特别地,在$v_2$看来,所描述的情景和图\ref{fig:unresolvable}b是一样的,后者$v_3$赞成$\bar a$且是良性行为的但回复较慢,而$v_1$是恶性行为的并在向$v_3$发送对$\bar a$的赞成票(因此导致$v_3$接受$\bar a$)的同时又向$v_2$发送对$a$的赞成票。

为了支持像{图\ref{fig:unresolvable}a}中协议层次的存活性概念,$v_1$需要一种方式来确保每个其它的完好节点最终能够在作用于$a$之前接受$a$。一旦是这种情形,则说系统在$a$上达成了一致就合情合理了。

\begin{definition}[认可]\label{def:agree}
	一个FBAS $\langle\mybm{V},\mybm{Q}\rangle$\textbf{认可}陈述$a$当且仅当无论接着会发生什么,一旦足够多的消息被发送且处理了,每个完好节点将接受$a$。
\end{definition}

\subsubsection{和中心化选举的比较}

为了理解上述问题为何会出现在联邦选举中,考虑一个由$N$个节点组成且{\quorum}大小为$T$的中心化拜占庭系统。这样一个系统在含不大于$f_L=N-T$个节点错误的情况下具有{\quorum}可达性。由于任意两个{\quorum}共享至少$2T-N$个节点,良性行为的节点的{\quorum}交持最多可容纳$f_S=2T-N-1$个拜占庭错误。

中心化拜占庭一致性系统通常设置$N=3f+1$,$T=2f+1$来使得$f_L=f_S=f$,该处的平衡点的安全性和存活性有相同的容错能力。如果安全性比存活性更为重要,一些协议增加$T$以使得$f_S>f_L$~\cite{Li:2007:BOF:1973430.1973440}。在FBA中由于{\quorum}有机成长,系统发现并使得自身处于平衡状态的可能性不大,这让在不谈存活性的情况下维持安全性变得更为重要。

现在考虑在中心化系统中一个节点$v$反对一个已被批准的陈述$a$的情形。如果$v$侦听到$f_S+1$个节点批准了$a$,$v$知道要么它们当中的一个是良性行为的,要么所有的安全性保证都崩溃了。不论哪种情形,$v$能在不损失安全性的条件下立即作用于$a$。FBA的等价的情景是侦听集合$B$,若$B$被删除则会损害良性节点的{\quorum}交。因为下面三个原因辨别出这样一个$B$是需要技巧的:第一,{\quorum}是动态被发现的;第二,恶性行为的节点可能会对切片撒谎;第三,$v$不知道哪个节点是良性行为的。取而代之的做法是,我们在{\vblock}接受$a$时定义联邦选举以接受$a$。{\vblock}属性有易于被检测的优势,但等价于在中心化系统中当我们确实需要$f_S+1$安全能力时侦听$f_L+1$个节点。

为了确保在中心化系统中获得所有良性行为节点的认可,只需要$f_L+f_S+1$个节点承认一个陈述被批准了。如果它们当中的多于$f_L$出现错误,我们不再期望存活性。如果$f_L$或更少的节点出现了错误,那么我们知道$f_S+1$个节点仍然愿意作证批准,这反过来说服所有其它的良性行为节点。然而$f_S$的可恢复性在FBA模型中也没有简单的对等情形。

换句话说,在某一时刻一个节点需要强烈信任某个陈述以依赖于它对安全性的真假。一个中心化的系统为陈述$a$提供两种途径来达到这个状态:直接批准$a$,或者从$f_S+1$个声明$a$被批准的节点上回溯推断,如果发现它们都撒谎了则认为安全性是不可期望的。FBA缺少后面一种方法;仅有的在良性行为节点上用于安全性的工具是直接批准。由于节点仍然需要有一种方式来克服给已被批准的陈述投反对票的情形,我们引入了接受的概念,但是它提供了一个限于完好节点的稍弱的相容性保障。

\subsection{陈述确认}

两类被接受陈述的局限性都来源于这一事实:一个完好的节点$v$可能会给一个已经被批准的陈述$a$投反对票。在反对$a$之后,它不会再投赞成票,这使得$v$将不可能批准$a$。为了给$v$提供一种在给$a$投反对票之后仍然能够批准$a$的方法,\textit{接受}的定义有一个基于{\vblock}的第二种准则。但是第二种准则弱于批准,对有{\quorum}交的被污染节点没有任何保障。

现在如果一个陈述$a$有从来没有任何完好的节点反对它这样一个属性,那么我们没有必要去接受它。完好的节点可以简单地批准$a$,而我们可以在作用于$a$之前要求它们这样做。我们成这样的陈述是不可驳斥的。

\begin{definition}[不可驳斥]
	当没有完好的节点可以反对一个陈述$a$时,称该条陈述在FBAS中是\textbf{不可驳斥的}。
\end{definition}

定理\ref{th8}告诉我们两个完好的节点不可能接受相互冲突的陈述。因此,尽管一些节点可能会反对某个被完好节点接受的陈述$a$,对\textit{一个完整的节点接受了$a$}这一陈述是不可反驳的。这暗示着举行第二次投票来批准一个完好的节点接受了$a$这一事实。

\begin{definition}[确认]
	在FBAS中一个{\quorum}$U_a$\textbf{确认}一个陈述$a$当且仅当$\forall \in U_a$, $v$声明承认$a$。一个节点\textbf{确认}$a$当且仅当它在这样一个{\quorum}中。
\end{definition}

节点通过声称``$accept(a)$''的接受陈述$a$,这是``一个完好的节点接受$a$''的缩略形式的陈述。一个良性行为的节点$v$仅当在接受了$a$之后才能投票赞成$accept(a)$,这是因为$v$不能假定其他的任何节点都是完好的。如果$v$本身是被污染的,$accept(a)$可能会失败,在这种情形写可能会牺牲$v$的存活性,然而不管怎样一个被污染的节点对存活性是没有任何保证的。

下一个定理表明节点能够依赖确认的陈述而不损失长最佳安全性。定理\ref{th10}则说明被确认的陈述符合第\ref{def:agree}中\textit{认可}的定义,这意味着节点可以依赖于确认的陈述而不会破坏完好节点的存活性。

\begin{theorem}\label{th9}
	设$\langle\mybm{V},\mybm{Q}\rangle$是一个满足除$B${\quorum}可交性的FBAS,且假定$B$包含了所有的恶性行为节点。令$v_1$和$v_2$是不在$B$中的两个节点。设$a$和$\bar a$是相互冲突的陈述。那么有若$v_1$确认了$a$,则$v_2$不会确认$\bar a$。
\end{theorem}

\begin{proof}
	首先注意到$accept(a)$和$accept(\bar a)$相互冲突——没有一个良性行为的节点会同时投票赞成两者。更进一步,注意到$v_1$必须批准$accept(a)$来确认$a$。由定理\ref{th5},$v_2$不会批准$accept(\bar a)$因此不会确认$\bar a$。
\end{proof}

\begin{theorem}\label{th10}
	如果一个含{\quorum}交的FBAS $\langle\mybm{V},\mybm{Q}\rangle$确认了一个陈述$a$,那么不论接下来会发生什么,一旦足够的消息被发送并处理了,每个完好的节点将会接受并确认$a$。
\end{theorem}

\begin{proof}
	令$B$是被污染节点的$DSet$,并令$U_a\not \subseteq B$是一个{\quorum},一个完好的节点通过它的确认$a$。令在$U_a\backslash B$中的节点广播$accept(a)$。根据定义,任何节点$v$,不论他是如何被投票的,在从一个{\vblock}集合中接受了$accept(a)$之后接受了$a$。因此,这些消息会说服额外的节点去接受$a$。让这些额外节点反过来广播$accept(a)$直到到达这样一个时刻:不论有多少更进一步的通信,没有更多的完好节点会接受$a$。一旦这个过程完成,设$V^+$是已经接受$a$的所有完好节点集合,设$V^-=(\mybm{V}\backslash B)\backslash V^+$是那些不能接受$a$的完好节点集。为证明所有的完好节点都接受了$a$,我们必须证明$V^-=\emptyset$。

	由定理\ref{th1},$U_a\backslash B$是$\langle\mybm{V},\mybm{Q}\rangle^{B}$中的一个{\quorum}。由于对于任何$v\in V^-$来说$V^+$都不是{\vblock}的,那么由定理\ref{th7}要么$V^-=\emptyset$要么$V^-$是$\langle\mybm{V},\mybm{Q}\rangle^{B}$中的一个{\quorum}。后一种情形导致这样一个矛盾:由于$V^-$只包含完好的(因此也是良性行为的)节点,它们当中的没有能够在首先真正接受$a$的情况下声明$accept(a)$,这意味着$U_a\backslash B\cap V^-=\emptyset$。然而这是不可能的,因为除$B$(一个$DSet$){\quorum}可交性告诉我们它的反面,即$U_a\backslash B\cap V^-\neq\emptyset$。	一旦每个节点都接受了$a$,所有的都会投票赞成确认$a$。因为完好的节点构成了一个{\quorum},这些投票将会成功。
\end{proof}

\todo{图11}总结了一个完好的节点为了确认$a$可以使用的路径。在不知情的状态下,$v$可能会赞成$a$或者与之冲突的$\bar a$。如果$v$投票赞成$\bar a$,它之后不会再赞成$a$,但是如果一个{\vblock}集合接受了$a$它却可以接受$a$。接下来确认消息的{\quorum}允许$v$去确认$a$,由定理\ref{th10}这意味着系统认可$a$。
\subsection{存活性和中和}\label{sec:vote_stuck}

不论中心化与否,分布式共识的主要挑战是:在系统对它的一致性认可之前,一个陈述可能卡在某个长期不确定的状态中。因此,一个协议不会尝试直接去批准外部表现的值\todo{语句不通}。一旦``{\slot}$i$处的值是$x$''这一陈述被阻塞了,系统将会永久性的地不能够在{\slot}$i$处达成一致,从而失去存活性。解决方案是仔细地巧妙修改投票中的陈述。在我们真正关心的问题上(即关于{\slot}内容)切断一个被阻塞陈述必然是可能的。我们称废弃一个被阻塞的陈述的过程为\textit{中和}。

\begin{figure}
\centering
\begin{tikzpicture}[thick,
    node/.style={draw,ellipse,inner xsep=-1.5mm,align=center,text width=2cm,
      inner ysep=0pt,
      text depth=1ex,text height=1em,fewshade=\maincolor},
    transition/.style={->,very thick, shorten <=-3pt, shorten >=0pt,
      align=center,font={\small\openup-.5\jot},draw=few-gray}
  ]
\node[node] (B) at (-2.5,0) {二价的};
\node[node] (AV) at (1.5,1.5) {$a$-价的};
\node[node] (NAV) at (1.5,-1.5) {$\na$-价的};
\node[node] (stuck) at (5.5,0) {被阻塞};
\node[node] (AA) at (5.5,1.5) {$a$~被同意};
\node[node] (NAA) at (5.5,-1.5) {$\na$~被同意};
\coordinate (merge) at ([xshift=-1cm] stuck.west);
\begin{scope}[on background layer]
\draw[transition] (B) to[out=90,in=180,looseness=.9] (AV);
\draw[transition] (B) to[out=270,in=180,looseness=.9] (NAV);
%\draw[transition] (AV) to[out=-2,in=90,looseness=.9] (stuck);
%\draw[transition] (NAV) to[out=2,in=-90,looseness=.9] (stuck);
\draw[transition] (AV) -- (AA);
\draw[transition] (NAV) -- (NAA);
\draw[transition] (AV) to[out=-60,in=180] (merge) -- (stuck);
\draw[transition,-] (NAV) to[out=60,in=180] (merge);
\draw[transition,-] (B) to (merge);
\end{scope}
\end{tikzpicture}
\caption{陈述$a$在系统范围内可能的状态}
\label{fig:system-states}
\end{figure}

更具体地将,{图\ref{fig:system-states}}描述了一个系统范围内一个陈述$a$潜在可能的状态。最初,系统是\textit{二价的}。我们用它来说明:存在一个可能的事件序列,所有的完好节点会接受$a$;而存在另一个序列,所有的完好节点会\textit{拒绝}$a$(即接受和$a$相冲突的陈述$\bar a$)。在某些时刻,两种输出的其中一个可能会不再可能。如果没有完好节点可以拒绝$a$,我们称它是$a-\!\!$价的;反之,如果没有节点可以接受$a$,我们称它是$\bar a-\!\!$价的。

在一个FBAS系统从二价态转化成$a-\!\!$价时,存在一种可能的结果:所有的完好节点都接受了$a$。然而这种情形并不总是成立。考虑像PBFT这样的四节点系统$\left\{v_1,\ldots,v_4\right\}$,其中任意三个节点构成了一个{\quorum}。如果$v_1$和$v_2$赞成$a$,系统变成了$a-\!\!$价的;没有哪三个节点可以批准一个相冲突的陈述。然而,如果$v_3$和$v_4$最终投票赞成了和$a$冲突的$\bar a$,它也不可能批准$a$。在这种情形,$a$的状态是永远不确定的,或称之为\textit{被卡住的}。

\begin{figure}
\centering 
\tabulinesep2pt
\begin{tabu} to .85\textwidth{ll}\toprule
\rowfont\bfseries 局部状态 & $a$系统范围的状态\\
\midrule
未提交 & 未知 (对任何节点) \\
已赞成~$a$ & 未知(对任何节点) \\
已赞成~$\na$ & 未知(对任何节点) \\
已接受~$a$ & 被阻塞, $a$-价的, or $a$ 被同意 \\
已确认~$a$ & $a$ 被同意 \\
\bottomrule
\end{tabu}
\caption{对于陈述$a$的状态一个完好节点所知道的信息}
\label{fig:localglobal}
\end{figure}

正如在{图\ref{fig:unresolvable}a}中所看到的那样,甚至一旦一个完好节点接受$a$,系统可能还是会在达到系统范围内对$a$的认可上失败。然而,根据定理\ref{thm:confirmed_stats_keep_liveness},一旦一个完好节点确认$a$,所有的完好节点最终都会接受它;因此系统对$a$认可。{图\ref{fig:localglobal}}总结了完好节点从它们自己关于一个陈述的局部状态可知的全局状态。

为了保持共识的可能性,一个协议必须确保每个陈述是不可辩驳因而不会被卡住,或者是可中和的因而如果被卡住不会阻塞进展。有两种流行的处理可中和陈述的方法:\textit{基于视图}的方法,由\textit{视图采样副本}~\cite{Oki:1988:VRN:62546.62549}最先提出并得到PBFT系统~\cite{Castro:1999:PBFT}的青睐;\textit{基于表决}的方法,由\textit{Paxos}~\cite{Lamport:1998:PP:279227.279229}发明。基于表决的方法可能更难以理解~\cite{Ongaro:2014:SUC:2643634.2643666}。更为混乱的是,人们经常称基于视图复制的方法为``Paxos算法''或者断言两种算法是相同的而事实上它们不是~\cite{6894199}。

基于视图的协议将投票中的{\slot}和单调递增的视图数联系相关联。万一在视图$n$处共识卡在了第$i$个{\slot}上,节点通过认可视图$n$包含少于$i$个有意义的{\slot}并且转向一个大一点的视图数。基于表决的方法将投票中的\textit{值}和单调递增的表决数相关联。万一一个表决被卡住了,节点用一个更大的表决数重新尝试,仔细选择和之前任何的不确定表决相冲突的值。

本文的工作采用了基于表决的方法,因为这样做更加简化了消除显著的基础节点(或称领导节点)的过程。例如存在赶上领导节点的行为~\cite{Lamport:2011:BPR:2075029.2075058}的可能。
