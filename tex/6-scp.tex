\section{SCP:一种联邦拜占庭协商协议}\label{sec:scp}

本节给出了恒星共识协议(Stellar Consensus Protocol,SCP)。高层次上来讲,SCP包含两个子协议:一个提名协议和一个表决协议。表决协议产生为槽点的候选值。如果运行时间足够长,它将在每个完好节点处产生相同的候选值集合,这意味着节点可以用一种确定性的方式把候选值组合起来。然而这有两大注意事项。首先,节点无法知道提名协议何时达到收敛点。其次,即使在收敛之后,恶性行为节点可能有能力重置提名过程多次。

当节点猜测到提名协议已经收敛后,它们执行表决协议,这采用了联邦选举来提交或终止与合成值相关联的表决。当节点同意提交一个表决时,和表决相关联的值将为讨论中的槽点具体化。当它们同意终止一个表决时,表决的值变得不再相关。如果一个表决在某个状态下被卡住了---在这种情形下一个或多个完好节点不能够提交或终止表决,那么节点重新使用更高的表决来尝试;他们将新的表决和被卡住的表决一样的值相关联,以防止任何节点错以为被卡住的节点被提交了。直观上来说,安全性来源于所有被卡住的或已提交的表决和相同的值相关联。存活性这一事实保证:一个被卡住的表决可以通过转向更高的表决来中立化。

本节的剩余部分给出了提名协议和表决协议。它们各自先以概念性语句描述,然后用带有表示概念性语句集合的消息的具体协议描述。最终,第\ref{sec:scp_correct}说明了协议的正确性。SCP认为每个槽点是完全独立的且可被看成是一个单一槽点共识协议中的许多分开的实例(和Paxos算法~\cite{Lamport:1998:PP:279227.279229}中的``单一法令议会''\footnote{single-decree synod。}类似)。我们必须总是在一个特定的槽的环境下解释例如候选值和表决等概念,而不像替他的讨论中隐式表述这样一个槽。

\subsection{提名协议}

因为仅仅要求{\slot}是偏序的,一些SCP的应用讲对每个{\slot}来说将只有一个貌似可信的表决。例如,在认证透明性中,每个中心认证机构可能有一系列它自己的{\slot}并且对每个{\slot}只签署唯一一份认证树。然而,其它一些应用承认一个{\slot}有多个貌似可信的值,这种情形对减少可能的输入值是有帮助的。我们的策略是从一个基于某种超时假设的可以获得共识的同步提名协议出发,并把提名协议的输出运用到一个安全性不依赖于时间的异步表决协议中~\cite{Lamport:2011:BAL:2075029.2075043}。这样一个初始的同步阶段有时候被称为调解者(conciliator,~\cite{Aspnes:2010:MAS:1835698.1835802})。

提名协议通过对于一个{\slot}的候选值趋于相同得以运作。节点然后确定性地将这些候选值组合起来成一个单独的对于{\slot}的\textit{合成}值。具体如何组合这些值取决于应用场景。举例而言,恒星网络使用SCP为每个{\slot}选择了一个事务集合和一个分类帐时间戳。为了组合这些候选值,恒星网络选择了这些事务集的并以及它们的时间戳的最大值。(含有无效的时间戳的值不会接受足够多的提名,从而不会变成候选者。)其它可能的方法包括,通过取集合交来组合集合,或者简单地选择那些有最大哈希值的候选者。

节点通过对陈述$nominate\;x$进行联邦选举产生出一个候选值$x$。

\begin{definition}[候选的]
	一个节点$v$认为一个值$v$是\textbf{候选值}当它确认了$nominate\;x$这样一个陈述——即$v$批准了$accept(nominate\;x)$。
\end{definition}

只要节点$v$没有候选值,$v$就有可能投票赞成$nominate\;x$,这里的$x$可以是任何通过了应用层有效性检验(如时间戳不能是未来时间点)的值。事实上,一般而言$v$应当重新提名它可观察到的其它的节点提名的任何值,不过会有一些接着将要谈到的防止候选者激增的限制。一旦$v$有了候选值之后,它必须停止对任何新的$x$的值的陈述$nominate\;x$进行投票。但是如联邦选举过程规定的那样,它仍然应当接受为新的值产生的$nominate$陈述(当被{\vblock}集合接受时),并确认新的$nominate$陈述。

当一个系统有完好的节点的时候(意味着它已避免了完全的错误)提名协议有一些属性。特别地,对每个{\slot}来说:
\begin{enumerate}
	\item\label{enum:cand_p1} 完好的节点至少可以产生一个候选值。
	\item\label{enum:cand_p2} 在某一时刻,可能的候选值集合停止增长。
	\item\label{enum:cand_p3} 如果任意一个完好的节点认为$x$是一个候选值的话,那么最终所有的完好节点都会认为$x$是一个候选值。
\end{enumerate}

现在考虑提名协议是如何达到这三个属性的。属性\ref{enum:cand_p1}可以满足是因为$nominate$陈述是不可辩驳的。节点从不会投票反对一个提名某个特别的值,且直到第一个候选值被确认为止完好节点可以提名任何值。只要有任何值通过了应用层的有效性检查,完好节点就可以投票赞成并确认$nominate\;x$。属性\ref{enum:cand_p2}可以保证,这是因为每个完好节点至少可以确认一个候选值——这会在一个有限时间内发生——不会有任何节点投票赞成新的值。因此,可能会成为候选值的只有那些已经被那些完好节点投票赞成的值。属性\ref{enum:cand_p3}是定理\ref{th10}的直接结果。

如果有少一点的值参与组合那么提名过程将会更为高效。因此,我们给节点指派一个暂时的优先级,并且如果可能的话让每个节点提名相同值的节点为更高优先级节点。更具体地讲,设$H$是一个加密哈希函数,其值域是一个整数集合$\left\{0,\dots,h_{max}-1\right\}$。($H$可能是$SHA-256$~\cite{shs2015},在这种情形下$h_{max}=2^{256}$。)令$G_i(m)=H(i, x_{i-1},m)$是一个特别用于{\slot}的哈希函数,这里的$x_{i-1}$是为$i$之前的{\slot}选用的值(或者当{\slot}是一个偏序时{\slot}$i$依赖的值的有序集合)。给定一个{\slot}$i$和一个轮计数值$n$,每个节点按照如下公式计算它的\textit{邻居}以及为它的每个邻居计算\textit{优先级}。

\begin{equation*}
	weight(v,v^{\prime}) = \frac{|\left\{q|q\in\mybm{Q}(v)\cap v^{\prime}\in q\right\}|}{|\mybm{Q}|}
\end{equation*}

\begin{equation*}
	neighbors(v,n) = \left\{v^{\prime}|G_i(N,n,v^{\prime})<h_{max}\cdot weight(v,v^{\prime})\right\}
\end{equation*}

\begin{equation*}
	priority(n,v^{\prime}) = G_i(P,n,v^{\prime})
\end{equation*}

$N$和$P$是生成两个不同哈希函数的常量。函数$weight(v,v^{\prime})$返回$\mybm{Q}$中包含$v^{\prime}$的切片的分数。通过考量$v^{\prime}$的权重作为概率的方式作用于$n$产生邻居$neighbors(v,n)$,我们也降低了不具很多信任的节点仍然可以支配某一轮的机会。

每个节点$v$最初应当找到一个节点$v_0\in neighbors(v,0)$使得它可以在它可以通信的节点间最大化$priority(0,v_0)$,然后投票赞成$nominate$和$v_0$相同的值。只有当$v=v_0$时$v$应当引入一个新的值来提名。$v$应当使用超时来决定新的用于投票赞成的$nominate$陈述。在超过时间限制$n$之后,$v$应当找到一个节点$v_n\in neighbors(v,n)$使得最大化$priority(n,v_n)$,且提名每个$v_n$所提名的。

\begin{theorem}\label{th11}
	最终所有的完好节点将有相同的合成值。
\end{theorem}

\begin{proof}
	这一定理的证明可由提名协议的三个属性而来。每个完好节点只会提名一个有限的表决。在没有恶性行为节点的情形下,完好节点将会在一个相同的候选值集合$Z$上收敛。为了阻止该收敛,恶性行为节点可能会引入新的候选值,在某个时间点上它可能是某些但不是所有的完好节点的候选者。这些候选者将需要获得从完好节点那里的选票,然而这使得它们是一个有限集合。最终恶性行为节点要么会停止扰乱系统,或者用光它们注入的新的候选值,在这种情形下完好的节点会在$Z$上收敛。
\end{proof}

\subsubsection{具体的提名协议}

\todo{图14}列出了一个节点$v$必须为每个{\slot}维护的提名协议状态。$X$是$v$投票赞成$nominate\;x$的值$x$的集合,$Y$是$v$接受的$nominate\;x$的集合,$Z$是候选值集合——即,所有的一个包含$v$的{\quorum}已经声明$accept(nominate\;x)$的值的集合。最后,$v$维护$N$——来自每个节点的最新值。(技术上来讲,$X$,$Y$和$Z$可以从$N$的值中重新算出来,但是能够直接引用它们的值是方便的。)所有的这些域最初都被初始化为空集。注意到$X$,$Y$和$Z$都总随时间的推移而增长——节点不会从这些集合中删除元素。

\todo{图15}说明了一个包含提名协议的具体消息。由于$X$和$Y$随时间单调递增,决定在不考虑网络发送顺序的情况下来自相同节点的多个\textsl{NOMINATE}消息哪个是最新的,只要$D$不改变中间提名(否则必须记录$D$的版本信息)。只有一个远程过程调用(RPC)对提名来说是必要的——参数是发送者的最新\textsl{NOMINATE}消息而返回值是接受者的(最新\textsl{NOMINATE}消息)。如果$D$或被提名的值是数字哈希值跑,必要时第二个RPC应当允许获取没有缓存的哈希原像。

由于节点无法通过任何手段知道何时提名协议是完全的,$SCP$必须在不同的节点处处理不同的合成值。一种优化方式则可以是,节点可以尝试在它们有候选值的之前尝试预测最终的合成值。为了这样做,合成只可以在$Z\neq \emptyset$时设为$combine(Z)$,否则在$Y\neq \emptyset$时设为$combine(Z)$,否则在$X\neq \emptyset$时设为$combine(X)$。这意味着高优先级节点可以在提名的同时初始化表决,在它的第一个\textsl{NOMINATE}消息上加上第一个表决消息\textsl{PREPARE}(表述参见下文)。
\subsection{表决协议}
一旦节点有了合成值它们将参与表决协议,尽管提名会继续更新合成值。一个表决$b$一个形如$b=\langle n,x\rangle$的二元组,这里$x\neq \perp$是一个值,而$b$是对讨论中的槽具体化的请示书(referendum)。$n\geq 1$是一个确保大些的表决数总是可访问的计数器。我们使用类C语言的标记$b.n$和$b.x$来表示表决$b$的计数和值的域,因而有$b=\langle b.n, b.x\rangle$。表决是全序的,而$b.n$比$b.x$更为重要{\footnote{译注:指二元组$b_1\prec b_2$当且仅当$b_1.n < b_2.n$或$b_1.n=b_2.n $且$b_1.x < b_2.x$;$b_1\equiv b_2$当且仅当$b_1.n=b_2.n\cap b_1.x=b_2.x$。}}为了方便起见,一个特殊无效的空表决$\mybm{0}=\langle 0,\perp\rangle$小于其它任何表决,而一个特别的计数器$\infty$大于其他所有的计数器。

我们分别用提交或终止一个表决$b$作为使用联邦投票来对语句$commib\;b$和$abort\;b$进行认可。对于给定的表决,$commit$和$abort$是相互冲突的,因此一个良性行为的节点最多为它们中的一个投赞成票。在\ref{sec:fba}的标注系统下,$commit\;b$的反是$\overline{commit\;b}$,但这里使用$abort\;b$更为直观。

由于对某个槽至多只有一个值被选用,所有提交的和被卡住的表决必须包含相同的值。粗略地说,这意味着如果陈述$commit$和更小的非终止表决相冲突的话那么它是无效的。

\begin{definition}[相容的]
	两个表决是\textbf{相容的}(记作$b_1 \sim b_2$)当且仅当$b_1.x=b_2.x$;它们是\textbf{不相容的}(记作$b_1\not\sim b_2$)当且仅当$b_1.x\neq b_2.x$。我们还将$b_1\leq b_2$(或等价地,$b_2\geq b_1$)且$b_1\sim b_2$记作$b_1\lesssim b_2$($b_2\gtrsim b_1$)。类似地,$b_1\lnsim b_2$或$b_2\gnsim b_1$意味着$b_1\leq b_2$(或等价地$b_2\geq \b_1$)且$b_1\not\sim b_2$。
\end{definition}

\begin{definition}[就绪的]
	一个表决$b$是\textbf{就绪的}当且仅当下面集合中的每个陈述都是正确的:$\left\{abort\;b_{old}|b_{odl}\lnsim b\right\}$。
\end{definition}

更准确地说,如果$b$被确认是就绪的话那$commit\;b$对投赞成票来说是有效的,节点通过在对应的终止陈述的联邦投票来保障它。全体一致地对这些陈述进行投票是方便的,因此不论我们在哪里写了``$b$就绪''周围的环境将应用于$abort$陈述的整个集合中。特别地,一个节点投票赞成、接受或确认$b$就绪当且仅当它分别投票赞成、接受或确认它们全部\textit{终止}了。

为了提交一个表决并向外界展示它的值$b.x$,SCP节点首先接受并确认$b$已经就绪,然后接受并确认$commit\;b$。在第一个完好节点投票赞成$commit\;b$之前,经由联邦投票的准备步骤确保所有完好节点最终可以确认$b$是就绪的。当一个完好节点$v$接受$commit\;b$时,意味着$b.x$最终将会被选中。然而,正如第\ref{sec:voting_safety}中所讨论的那样,为了防止$v$被污染$v$必须在作用于它之前确认$commit$。

\subsubsection{具体的表决协议}\label{sec:scp_ballot_concrete}

\todo{图16}强调了由每个节点维护的每一{\slot}的状态。一个节点$v$存储了:它当前的表决$b$;两个最近的已经认定就绪的且不相容的表决对$(p,p^{\prime})$;它必须投票\textit{提交}的(或在后续阶段需要确认\textit{提交}的)最小表决$c$(如果存在的话),对此它还没有接着接受到\textit{终止类}陈述;已确认就绪的最高表决$P$;从每个节点($M$)处接受到的最新消息;以及状态$\varphi$。表决$b$,$p$,$p^{\prime}$和$P$在同一个阶段里是不减的。另外,如果$c\neq\mybm{0}$——意味着$v$可能参与了批准$commit\;c$——代码必须确保$c\lesssim P\lesssim b$。这一不变量保证了节点总是可以投票为当前的表决$b$做好准备。

\todo{图17}展示了协议消息。注意$a\;\vee accept(a)$是每个节点需要为一个{\quorum}所断言的,使得它们按照\textit{接受}定义中的第\ref{itm:cond_normal}种方式接受$a$。每个节点通过设置$b\leftarrow \langle 1,combine(Z)\rangle$,$p\leftarrow \mybm{0}$,$p^{\prime}\leftarrow \mybm{0}$,$P\leftarrow \mybm{0}$,$c\leftarrow \mybm{0}$,$M\leftarrow\emptyset$及$\varphi \leftarrow \textsl{PREPARE}$的方式初始化{\slot}的状态。之后节点在同类间重复地交换消息,发送由 $\varphi$表明的任何消息。一旦给$M$添加了一个新近接受的消息,一个节点$v$按照下面的方式添加它的状态:

\begin{enumerate}\label{protocal_case}
	\item 如果$\varphi = \textsl{PREPARE}$且接受的信息让$v$接受新表决是就绪的,更新$p$和$p^{\prime}$。之后,如果$c\neq \mybm{0}$且$p\gnsim P$或$p^{\prime}\gnsim P$,设置$c\leftarrow \mybm{0}$。
	\item 如果$\varphi = \textsl{PREPARE}$且$v$确认新表决是就绪的,增加$P$。之后,如果$c=\mybm{0}$,$P\geq b$,且$p\gnsim P$或$p^{\prime}\gnsim P$都不成立,则设$c\leftarrow P$且$b\leftarrow P$(尽管通常$b=P$已经成立)。
	\item 如果$\varphi = \textsl{PREPARE}$,且$v$接受一个或多个相容表决的\textit{提交类}消息。设$c$为最小的这类表决,$P$设为最大的使得``$v$能够接受所有的$\left\{commit\;b^{\prime}|c\lesssim b^{\prime}\lesssim P\right\}$,$b\leftarrow \langle \infty, c.x\rangle$和$\varphi\leftarrow \textsl{CONFIRM}$''的表决。
	\item 如果$\varphi = \textsl{CONFIRM}$,且接受的消息让$v$接受新表决为就绪的,则提升$p$至最高已被接受为就绪态的、且满足$p\sim c$的表决。
	\item 如果$\varphi = \textsl{CONFIRM}$且$v$接受更多的相容的\textit{提交类}消息,提升$p$至最高的``使得$v$接受所有的$\left\{commit\;b^{\prime}|c\lesssim b^{\prime} \lesssim P\right\}$''的表决。
	\item 如果$\varphi = \textsl{CONFIRM}$且$v$对任意$c^{\prime}$确认$commit\;c^{\prime}$,设置$c$和$P$为最低和最高的这类表决,设$\varphi\leftarrow\textsl{EXTERNALIZE}$,具体化$c.x$并结束。
\end{enumerate}

当$c=\mybm{0}$时,上述协议实施联邦选举来确认$b$已经就绪。一旦$c\neq \mybm{0}$,该协议对$commit\;c$ (实际上是介于$c$和$P$之间的相容的表决)实施联邦选举。对确认阶段来说,一旦一个良性行为的节点$v$接受了$commit\;c$,该节点就不会接受或尝试确认任何满足$c^{\prime}\not\sim c$的$commit\;c^{\prime}$。因此直观上说,一旦一个\textit{提交}被确认了,只要节点具有{\quorum}交属性那么具体化它的值就是安全的。

所有来自一个节点的消息在元组$\langle \varphi,b,p,p^{\prime},P\rangle$的定义之下是全序的,这里$\varphi$是最重要的域而$P$最不重要。所有的\textsl{PREPARE}消息都在\textsl{CONFIRM}消息之前,转而对于给定的{\slot}来说在单独的\textsl{EXTERNALIZE}消息之前。\textsl{PREPARE}信息显式地包含这四个域,而\textsl{CONFIRM}和\textsl{EXTERNALIZE}包含\todo{图17}中所描述的值。这一序关系使得$M$值包含来自每个节点的最新表决而不依赖于时间来排序消息成为可能,这是因为网络环境可能会对消息重新排序。

一些协议的细节需要解释。形如``$abort\;b^{\prime}\vee accpet(abort\;b^{\prime})$''的由\textsl{PREPARE}所蕴含的陈述并没有指明$v$是否赞成或确认$abort\;b^{\prime}$。对于\textit{接受}的定义来说这种区分并不重要。掩盖这种区分使得$v$忘记了旧的它投票提交(因此不能够投票终止)的表决--只要它为这些表决接受一个\textit{终止类}消息的话。

为了确保节点收敛于$P$,$p$和$p^{\prime}$都是必需的,这是因为定理\ref{thm:confirmed_stats_keep_liveness}要求节点重新广播它们已经接受的消息。从\textit{就绪}的定义可知,``一个节点接受为就绪态的、两个不相容的最高表决''蕴含了``所有该节点接受为就绪态的表决''。

在$v$发出\textsl{ENTERNALIZE}消息的时候它实际上已经接受了一个区间内的\textit{提交类}消息$\left\{commit\;b^{\prime}|b^{\prime}\gtrsim c\right\}$。然而,$v$设置$P$来断言``只有它确认提交的表决是可接受的'',而不是在隐式的\textsl{CONFIRM}消息中设置$P.n=\infty$从而对每个$b^{\prime}\gtrsim c$断言$accept(commit\;b^{\prime})$。这样做是足够的,因为一旦一个单独的完好节点确认了$commit\;c$,定理\ref{thm:confirmed_stats_keep_liveness}告诉我们所有的完好节点也将确认它。把关注点集中在已被确认的表决上有额外的好处:\textsl{EXTERNALIZE}消息仅断言$v$已经批准的信息,从而使得$\mybm{Q}(v)$不再相关。这意味着一个独立静态的\textsl{EXTERNALIZE}消息对未来任意远处的想赶上进度的节点来说都是有用的,即使{\quorum}切片与此同时已经改变了很多。

交换表决消息只需要一个RPC。参数是发送者最新的消息而返回值是接受者最新的消息。对于\textsl{NOMINATE},如果$D$或在表决中的值$x$是加密哈希,那么为了取回没有被缓存的哈希原像需要一个单独的RPC。
\subsubsection{表决选择}

如果所有的节点都设置$b$为同样的表决,那么在\todo{页xxx}的第1-6步\todo{xxx}就足够认可表决的值并且具体化它了。然而,在表决协议开始时提名协议并不一定已经在每处都产生了相同的值,在这种情形下节点可能会在它们尝试提交第一个表决的时候失败。如果一个表决失败了或花费了太多的时间那么它可能因为没有回复的节点而失败,那么该节点必须增加它的表决计数器而用一个更高的表决来重新尝试。

当专用新的表决时,节点$v$设置$b_v\leftarrow \langle b_v.n+1,z\rangle$,这里$z$是由这个决定的:如果$P\neq \mybm{0}$,那么$z=P.x$;否则,$z$是合成值$combine(Z)$,这里$Z$是第\ref{sec:scp_nominate_concrete}节提到的候选值集。由于$c$总是由$P=b$初始化而来,提高$P.x$的优先级确保在每个节点处的不变量:如果$c\neq \mybm{0}$那么$c\lesssim P\lesssim b$。

方便起见,我们用下标来区分属于特殊节点或消息的域。如果$v$是一个节点,那么用$b_v$,$p_v$,$p_v^{\prime}\ldots$表示\todo{图16}中描述的节点$v$的状态里的$b,p,p^{\prime}\ldots$。我们还设置$z_v$为上文提及的$v$的下一个表决---即,在$P\neq \mybm{0}$时$z_v=P_v.x$否则为$v$的合成值。类似地,令$b_m,p_m,p^{\prime}_m,P_m,c_m$表示在\todo{图17}中所描述的由一个网络消息$m$所包含的对应的域。

\begin{definition}[自我验证]
	在节点$v$处的消息集合$S\subseteq M_v$是\textit{自我验证}的,当$v$是$S$中的某个发送者而$S$的发送者集合是一个在消息的$D$域下声明的{\quorum}集合之下的一个{\quorum}。
\end{definition}

一个节点应当在$\varphi=\textsl{PREPARE}$且下面条件之一满足的情况下放弃当前的表决$b_v$而转向一个更高的表决:

\begin{enumerate}
	\item\label{enum:giveup1} $M_v$包含一个来自发送者的{\vblock}集合的消息集合$S$,使得$\forall m\in S,b_v.n<b_m.n$且要么$b_m.n=\infty$要么$P_v\lesssim c_m$(意味着$v$的表决计数器已经落后了)。
	\item\label{enum:giveup2} 自从$M_v$第一次包含一个自我验证的集合$S$(满足$\forall m\in S,b_v.n<b_m.n$)之后已经超时过期了(意味着$b_v$很可能不会再提交,且$v$希望它已经收到了足够多的消息来让它选择一个更好的新的表决)。
\end{enumerate}

为了让协议最终能够终止,条件\ref{enum:giveup2}必须最终足够长以使得所有良性行为节点能够交换几轮消息。为了在不需要预测网络延迟的情形下达到这个具体的实现,超时值应当设置为随着$b_v.n$的增长而增长。同样需要注意当节点$v$已经落后时,并不是遍历测试每个计数器,而是可以简单地增加$b_v.n$到最小值来使得条件\ref{enum:giveup1}为假。
\subsection{正确性}\label{sec:scp_correct}

一个节点只有在已经许诺确认所有小编号的表决的\textit{终止}陈述之后才能担保确认$commit\;b$陈述。因为一个良性行为的节点不能够接受(因此也不担保确认)相冲突的陈述,这意味着对于给定的$\mybm{V},\mybm{Q}$,定理\ref{th5}确保一个良性行为节点集合$S$只要享有除$\mybm{V}\backslash S${\quorum}可交性则不会产生相互冲突的值。如果$\mybm{V}$和$\mybm{Q}$只在{\slot}间改变的话那么安全性仍然成立,但如果它们在{\slot}中(mid-slot)改变呢(例如用于应对节点崩溃)?为了分析在重新配置的情形下的安全性,我们保守地对旧的和新的{\quorum}切片集合进行交操作;这反映了这样一个事实:节点可能依据来自不同时期的消息的组合来作出决定。因此很保守地讲,一个节点只有在当前{\slot}用到的每个配置下都是完好的我们才说它是完好节点。但是我们可以放松要求而说:如果一个节点在最近的配置中都是完好的并且在以往的配置中从未接受过来自全部由恶性行为节点的{\vblock}集合发来的消息,则我们称该节点是完好的。

\begin{theorem}\label{th12}
        令$\langle \mybm{V_1},\mybm{Q_1}\rangle,\ldots,\langle \mybm{V_k},\mybm{Q_k}\rangle$是一个FBAS在协商一个单独{\slot}的时候经历过的配置集合。令$\mybm{V}=\mybm{V_1}\cup \cdots\cup \mybm{V_k}$且$\mybm{Q}(v)=\left\{q|\exists j, v\in\mybm{V_j}\cap q\in\mybm{Q}_j(v)\right\}$。令$B\subseteq\mybm{V}$是一个集合,满足$B$包含所有已经发送了非法消息的恶性行为节点——尽管$\mybm{Q}\backslash B$可能仍然包含崩溃(不响应)的节点。假设$v_1\not\in B$具体化了$x_1$,而$v_2\not\in B$具体化了$x_2$。则如果$\langle\mybm{V},\mybm{Q}\rangle^{B}$有{\quorum}交,那么$x_1=x_2$。
\end{theorem}

\begin{proof}
        为了让$v_1$产生具体的$x_1$,它必然已经和一个伪{\quorum}$U_1\subseteq\mybm{V}$合作批准了$accept(commit(\langle n_1,x_1\rangle))$。我们称伪{\quorum}是因为$U_1$对任意特殊的$j$来讲可能都不是$\langle\mybm{V}_j,\mybm{Q}_j\rangle$的{\quorum},这是由于批准可能已经涉及包含多个配置的消息。然而,为了使得批准成功,$\forall v\in U_1,\exists j, \exists q\in\mybm{Q}_j(v)$使得$q\subseteq U_1$。从$\mybm{Q}$的构造方式可知$q\in\mybm{Q}$。因此$U_1$是$\langle\mybm{V},\mybm{Q}\rangle$。类似地,一个伪{\quorum}必然已经批准了$accept(commit\langle n_2,x_2\rangle)$,且$U_2$一定是$\langle\mybm{V},\mybm{Q}\rangle$的一个{\quorum}。根据$\langle\mybm{V},\mybm{Q}\rangle^{B}$的{\quorum}交,必然存在某个$v\in \mybm{V}\backslash B$使得$v\in U_1\cap U_2$。根据假设,这样的$v\not\in B$不能声明接受不相容的表决。由于$v$确认接受值为$x_1$和$x_2$的表决提交,那么必然有$x_1=x_2$。
\end{proof}

对于节点$v$的存活性,当一个FBAS对一个单独节点经历了一系列的重新配置$\langle \mybm{V_1},\mybm{Q_1}\rangle,\ldots,\langle \mybm{V_k},\mybm{Q_k}\rangle$时我们需要考虑一些东西。首先,定理\ref{th12}的安全性前提条件必须对$v$以及$v$关心的节点成立,因为违反安全性会破坏定理\ref{th10}中所要求的{\quorum}交属性。其次,最新状态下的恶性行为节点集合$\langle \mybm{V_k}, \mybm{Q_k}\rangle$必须不能是{\vblock}的,这是因为这会否定一个{\quorum}而阻止它批准陈述。最后,$v$的状态不能够被一个{\vblock}集合错误地声称接受$\langle \mybm{V_1},\mybm{Q_1}\rangle\cdots \langle \mybm{V_{k-1},\mybm{Q_{k-1}}\rangle}$的中一个陈述所污染。总之,我们认为节点$v$在下面的条件成立的情况下是完好的。首先,在恶性行为节点被删除时对当前{\slot}的过去的配置的并必须有{\quorum}交。其次,$v$必须在最新的视图$\langle \mybm{V_k}, \mybm{Q_k}\rangle$依据以往的静态配置标准还是完好的。最后,$v$必须从未接受过以往的配置$\langle \mybm{V_1},\mybm{Q_1}\rangle\cdots \langle \mybm{V_{k-1},\mybm{Q_{k-1}}\rangle}$中的全由恶性行为节点组成的{\vblock}集合中接受过消息。

\begin{theorem}\label{th13}
        在一个有{\quorum}交的FBAS中,如果一个表决计数器为$b.n=n\neq \infty$的完好节点开始条件\ref{enum:giveup2}中的计时器,那么所有的节点能够将它们的表决计数器提升至$n$而不需要任何计时器过期。
\end{theorem}

\begin{proof}
        一个节点只有在它看到了来自{\quorum}$U$的消息时才开始计时,这是每个$U$中的成员都有一个等于或大于$n$的计数值。如果$U$中包含了所有的完好节点,结论已然成立。否则,令$U^{\prime}\neq \emptyset$是$U$的完好节点子集。根据除被污染$DSet${\quorum}可交性,比存在至少一个完好节点$v\not\in U$使得$U^{\prime}$是$\vblock$的。(否则定理\ref{th10}的证明给出矛盾。)
        
	如果$\forall v^{\prime}\in U^{\prime},\varphi=\textsl{PREPARE}$,意味着$\forall v^{\prime}\in U^{\prime},b_{v^{\prime}}.n\neq \infty$,那么由\todo{页xxx}的第\ref{enum:giveup1}个条件,$v$将会简单地提升$b_v$直到$b_v.n\geq n$。否则,假设$\exists v^{\prime}\in U^{\prime}$使得$\varphi_{v^{\prime}}=\textsl{PRIVATE}$。这意味着$v^{\prime}$(或其他完好节点)确认$c_{v^{\prime}}$就绪了且接受了$commit\;c_{v^{\prime}}$。因此,由定理\ref{th10},$v$也将在它接受到足够多的其他完好节点的消息之后确认$c_{v^{\prime}}$是就绪的。此外,根据定理\ref{th8}任何完好节点都不会接受任何$b^{\prime}\gnsim c_{v^{\prime}}$的表决为就绪的。这意味着一旦$P_v\gtrsim c_{v^{\prime}}$成立,它将永远继续成立下去,再次由于条件\ref{enum:giveup1}$v$将增加$b_v$直到$b_v.n\geq n$。
        
        既然$b_v.n\geq n$,如果完好节点中还有一个小的表决计数器那么在前两个段落中提到的过程将至少提升这样节点中的一个,这同样要求没有超时限制。这个过程将会重复直到翁恩完好的节点的表决计数器都大于等于$n$。
\end{proof}

\begin{theorem}\label{th14}
        在一个有{\quorum}交且有足够时间供完好节点来交换足够多的消息的FBAS中,所有的完好节点最终将会产生一个值。
\end{theorem}

\begin{proof}
        根据定理\ref{th14},所有的节点最终将会有候选值的相容集合$Z$。假设已经经历了这个时刻并且每个完好节点都有相同的合成值$z=combine(Z)$。如果没有节点在$b.x=z$不满足的情形下从未确认任何表决$b$是就绪的,那么所有新的完好节点的表决将会有值$z$。令$b=\langle n,z\rangle$为任何完好节点中的最高的含$z$表决。最终其他节点将会超时而赶上$b$,直到一个{\quorum}赶上而开始设置新的计时器。如果时间足够长,那么根据定理\ref{th13}其他节点将会赶上而它们将会完成协议,确认$commit\;b$。
        
        否则,给定足够长的时间,根据定理\ref{th10}最终完好节点将收敛于相同的$P$。这时,如果我们将$z=combine(Z)$替换为$z=P.x$的话那么和前段中一样的参数也会成立。
\end{proof}

值得注意的是如果我们去除了``足够超时限制''这一条件的话那么定理\ref{th14}不再为真。特别地,存在这样一个时间段,一个完好节点集合已经进行了足够多的投票来确认一个表决就绪了,而此时此刻节点意识到该表决确实确认就绪了。在有足够长的消息延迟的情况下,节点可以潜在地在不同的$P.x$值之间选择——即在任何节点意识到$P$已经确认就绪了之前,每个节点已经前移了并投票确认$P^{\prime}\gnsim P$已经就绪。根据著名的不可能性结果~\cite{Fischer:1985:IDC:3149.214121},一些持久性抢占情景是不可避免的。因此,我们最可能希望的是在一个半异步(partial synchrony)~\cite{Dwork:1988:CPP:42282.42283}的假设下的存活性,这正式定理\ref{th14}给我们保证的。